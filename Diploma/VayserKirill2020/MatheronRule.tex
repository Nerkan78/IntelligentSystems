\documentclass[12pt, twoside]{article}
\setcounter{secnumdepth}{4}
\usepackage[utf8]{inputenc}
\usepackage[english,russian]{babel}
\newcommand{\hdir}{.}
\usepackage{graphicx}
\usepackage{caption}
\usepackage{amssymb}
\usepackage{amsmath}
\usepackage{amsthm}
\usepackage{mathrsfs}
\usepackage{euscript}
\usepackage{upgreek}
\usepackage{array}
%\usepackage{theorem}
\usepackage{graphicx}
\usepackage{subfig}
\usepackage{caption}
\usepackage{color}
\usepackage{url}
\usepackage{cite}
\usepackage{geometry}
\usepackage{tikz,fullpage}
\usepackage{enumerate}
\usepackage{autonum}
\usepackage{enumitem}%
%\usepackage[unicode, pdftex]{hyperref}
\usepackage{comment}
\usepackage{titlesec}
\usepackage{multirow}
\usepackage{hyperref}
\titleformat{\paragraph}[runin]{\normalfont\normalsize\bfseries}{}{0pt}{}
% объявляем новую команду для переноса строки внутри ячейки таблицы
\newcommand{\specialcell}[2][c]{%
  \begin{tabular}[#1]{@{}c@{}}#2\end{tabular}}
\newcommand{\bmatr}{{\mathbf{B}}}
\newcommand{\cmatr}{{\mathbf{C}}}
\newcommand{\hmatr}{{\mathbf{H}}}
\newcommand{\fmatr}{{\mathbf{F}}}
\newcommand{\mmatr}{{\mathbf{M}}}
\newcommand{\xmatr}{{\mathbf{X}}}
\newcommand{\pmatr}{{\mathbf{P}}}
\newcommand{\xmatrt}{{\tilde{\mathbf{X}}}}
\newcommand{\imatr}{{\mathbf{I}}}
\newcommand{\vmatr}{{\mathbf{V}}}
\newcommand{\wmatr}{{\mathbf{W}}}
\newcommand{\umatr}{{\mathbf{U}}}
\newcommand{\zmatr}{{\mathbf{Z}}}
\newcommand{\zmatrt}{{\tilde{\mathbf{Z}}}}
\newcommand{\Tmatr}{\mathbf{T}}
\newcommand{\lambdamatr}{{\mathbf{\Lambda}}}
\newcommand{\phimatr}{\mathbf{\Phi}}
\newcommand{\sigmamatr}{\mathbf{\Sigma}}
\newcommand{\thetamatr}{\boldsymbol{\Theta}}
\newcommand{\Ri}{\mathcal{R}}
\newcommand{\rk}{\mathfrak{r}}
\newcommand{\ab}{{\wm}}
\newcommand{\bA}{{\mathbf{A}}}
\newcommand{\ba}{{\mathbf{a}}}
\newcommand{\bb}{{\mathbf{b}}}
\newcommand{\cb}{{\mathbf{c}}}
\newcommand{\db}{{\mathbf{d}}}
\newcommand{\eb}{{\mathbf{e}}}
\newcommand{\fb}{{\mathbf{f}}}
\newcommand{\gb}{{\mathbf{g}}}
\newcommand{\hb}{{\mathbf{h}}}
\newcommand{\mb}{{\mathbf{m}}}
\newcommand{\pb}{{\mathbf{p}}}
\newcommand{\qb}{{\mathbf{q}}}
\newcommand{\rb}{{\mathbf{r}}}
\newcommand{\tb}{{\mathbf{t}}}
\newcommand{\ub}{{\mathbf{u}}}
\newcommand{\vb}{{\mathbf{v}}}
\newcommand{\wb}{{\mathbf{W}}}
\newcommand{\xb}{{\mathbf{x}}}
\newcommand{\bz}{{\mathbf{z}}}
\newcommand{\bZ}{{\mathbf{Z}}}
\newcommand{\xt}{{\tilde{x}}}
\newcommand{\xbt}{\tilde{{\mathbf{x}}}}

\newcommand{\Kb}{{\mathbf{K}}}
\newcommand{\Xb}{{\mathbf{X}}}
\newcommand{\Ib}{{\mathbf{I}}}
\newcommand{\yb}{{\mathbf{y}}}
\newcommand{\zb}{{\bz}}
\newcommand{\zt}{{\tilde{z}}}
\newcommand{\zbt}{{\tilde{\bz}}}
\newcommand{\mub}{{\boldsymbol{\mu}}}
\newcommand{\alphab}{{\boldsymbol{\alpha}}}
\newcommand{\thetab}{{\boldsymbol{\theta}}}
\newcommand{\iotab}{\boldsymbol{\iota}}
\newcommand{\zetab}{\boldsymbol{\zeta}}
\newcommand{\xib}{\boldsymbol{\xi}}
\newcommand{\xibt}{\tilde{\boldsymbol{\xi}}}
\newcommand{\xit}{\tilde{\xi}}
\newcommand{\betab}{{\boldsymbol{\beta}}}
\newcommand{\phib}{{\boldsymbol{\phi}}}
\newcommand{\psib}{{\boldsymbol{\psi}}}
\newcommand{\gammab}{{\boldsymbol{\gamma}}}
\newcommand{\lambdab}{{\boldsymbol{\lambda}}}
\newcommand{\varepsilonb}{{\boldsymbol{\varepsilon}}}
\newcommand{\pib}{{\boldsymbol{\pi}}}
\newcommand{\sigmab}{{\boldsymbol{\sigma}}}
\newenvironment{comment}{}{}



\newcommand{\scl}{s_{\mathsf{c}}}
\newcommand{\shi}{s_{\mathsf{h}}}
\newcommand{\shib}{\mathbf{s}_{\mathsf{h}}}
\newcommand{\MOD}{M}
\newcommand{\entr}{\mathsf{H}}
\newcommand{\REG}{\Omega}
\newcommand{\Mquol}{V}
\newcommand{\prob}{p}
\newcommand{\expec}{\mathsf{E}}

\newcommand{\xo}{{\overline{x}}}
\newcommand{\Xo}{{\overline{x}}}
\newcommand{\yo}{{\overline{y}}}

\newcommand{\xbo}{{\overline{\mathbf{x}}}}
\newcommand{\Xbo}{{\overline{\mathbf{X}}}}

\newcommand{\Amc}{{\mathcal{A}}}
\newcommand{\Bmc}{{\mathcal{B}}}
\newcommand{\Cmc}{{\mathcal{C}}}
\newcommand{\Jmc}{{\mathcal{J}}}
\newcommand{\Imc}{{\mathcal{I}}}
\newcommand{\Kmc}{{\mathcal{K}}}
\newcommand{\Lmc}{{\mathcal{L}}}
\newcommand{\Mmc}{{\mathcal{M}}}
\newcommand{\Nmc}{{\mathcal{N}}}
\newcommand{\Pmc}{{\mathcal{P}}}
\newcommand{\Tmc}{{\mathcal{T}}}
\newcommand{\Vmc}{{\mathcal{V}}}
\newcommand{\Wmc}{{\mathcal{W}}}
\newcommand{\Smi}{{\mathcal{S}}}
\newcommand{\Xmc}{{\mathcal{X}}}

\newcommand{\Rbb}{{\mathbb{R}}}


\newcommand{\T}{^{\text{\tiny\sffamily\upshape\mdseries T}}}
\newcommand{\deist}{\mathbb{R}}
\newcommand{\ebb}{\mathbb{E}}

\newcommand{\Amatr}{\wm}
\newcommand{\X}{\mathbf{X}}
\newcommand{\Z}{\mathbf{Z}}
\newcommand{\Umatr}{\mathbf{U}}
\newcommand{\zetavec}{\boldsymbol{\zeta}}

\newcommand{\M}{\mathbf{M}}
\newcommand{\x}{{\mathbf{x}}}
\newcommand{\z}{{\bz}}
\newcommand{\ical}{{\mathcal{I}}}
\newcommand{\tvec}{{\mathbf{t}}}
\newcommand{\xvec}{{\mathbf{x}}}
\newcommand{\zvec}{{\bz}}
\newcommand{\bvec}{{\mathbf{b}}}
\newcommand{\qvec}{{\bz}}
\newcommand{\pvec}{{\mathbf{p}}}
\newcommand{\wvec}{{\mathbf{W}}}
\newcommand{\rvec}{{\mathbf{r}}}
\newcommand{\thetavec}{{\mathbf{\theta}}}
\newcommand{\y}{{\mathbf{y}}}
\newcommand{\g}{{\mathbf{g}}}
\newcommand{\w}{{\mathbf{W}}}
\newcommand{\wm}{{\mathbf{w}}}
\newcommand{\m}{{\mathbf{m}}}


\newtheorem{theorem}{Теорема}
\newtheorem{corollary}{Утверждение}
\newtheorem{definition}{Определение}
\DeclareMathOperator*{\argmin}{arg\,min}
\graphicspath{ {pics/} }
%\graphicspath{ {fig/} }
\begin{document}
\selectlanguage{russian}
\title{\textbf{Правило Маферона}}
\date{}
\maketitle


Работа посвящена эффективному сэмплированию из гауссовских процессов. Стандартная схема с репараметризацией требует минимум $O(m^3)$ операций для сэмплирования $m$ значений. Было разработано несколько методов представления гауссовского процесса для увеличения скорости сэмплирования. Ключевой идеей является разложение апостериорного распределения в сумму двух слагаемых~---априорного и некоторого обновления распределения. Представление этих слагаемых в определенном виде позволяет увеличить эффективность сэмлирования. Теоретическим обоснованием возможности подобного разложения служит предлагаемая к рассмотрению теорема.  
\begin{theorem}
(Правило Маферона). Пусть $\ba$ и $\bb$ - имеют совместное нормальное многомерное распределение. Тогда условное распределение на $\ba$ при $\bb = \mathbf{\beta}$ вычисляется как 
\begin{equation}
(\boldsymbol{a} \mid \boldsymbol{b}=\boldsymbol{\beta}) \stackrel{\mathrm{d}}{=} \boldsymbol{a}+\operatorname{Cov}(\boldsymbol{a}, \boldsymbol{b}) \operatorname{Cov}(\boldsymbol{b}, \boldsymbol{b})^{-1}(\boldsymbol{\beta}-\boldsymbol{b})
\end{equation}

\begin{proof}
Доказательство можно провести двумя способами. Первый носит сугубо технический характер и представляет собой прямое вычисление матожидания и ковариационных матриц обоих частей по определению. В силу свойств нормального распределения, это будет означать совпадение функций плотности. Подробные вычисления читатель может увидеть 
\href{http://fourier.eng.hmc.edu/e161/lectures/gaussianprocess/node7.html}{здесь}

Мы приведем более изящное доказательство. Для удобства проведем переобозначения : $\Sigma_{11} = \operatorname{Cov}(\ba, \ba), \; \Sigma_{12} = \operatorname{Cov}(\ba, \bb), \; \Sigma_{21} = \operatorname{Cov}(\bb, \ba), \; \Sigma_{22} = \operatorname{Cov}(\bb, \bb) $. Пусть также $E \ba = \mub_1, \; E \bb = \mub_2$

Обозначим $\bz=\ba+\bA \bb,$ где $\bA=-\Sigma_{12} \Sigma_{22}^{-1} .$ Тогда
$$
\begin{aligned}
\operatorname{Cov}\left(\bz, \bb\right) &=\operatorname{Cov}\left(\ba, \bb\right)+\operatorname{Cov}\left(\bA \bb, \bb\right) =\Sigma_{12}+\bA \operatorname{var}\left(\bb\right) \\
&=\Sigma_{12}-\Sigma_{12} \Sigma_{22}^{-1} \Sigma_{22}=0
\end{aligned}
$$
Это значит, что $\bz$ и $\bb$ некореллированны и, вследствие их нормальности, независимы. Тогда $E\bz=\boldsymbol{\mu}_{1}+\bA \boldsymbol{\mu}_{2}.$ Из этого следует, что
$$
\begin{aligned}
E\left(\ba \mid \bb\right) &=E\left(\bz-\bA \bb \mid \bb\right) \\
&=E\left(\bz \mid \bb\right)-E\left(\bA \bb \mid \bb\right) \\
&=E(\bz)-\bA \bb \\
&=\boldsymbol{\mu}_{1}+\bA\left(\boldsymbol{\mu}_{2}-\bb\right) \\
&=\boldsymbol{\mu}_{1}+\Sigma_{12} \Sigma_{22}^{-1}\left(\bb-\boldsymbol{\mu}_{2}\right)
\end{aligned}
$$
Получается, что матожидания левой и правой части искомого тождества совпадают. Осталось рассмотреть ковариацию:
$$
\begin{aligned}
\operatorname{var}\left(\ba \mid \bb\right) &=\operatorname{var}\left(\bz-\bA \bb \mid \bb\right) \\
&=\operatorname{var}\left(\bz \mid \bb\right)+\operatorname{var}\left(\bA \bb \mid \bb\right)-\bA \operatorname{Cov}\left(\bb,\bz\right)-\operatorname{Cov}\left(\bz,\bb\right) \bA^{\top} \\
&=\operatorname{var}\left(\bz \mid \bb\right) \\
&=\operatorname{var}(\bz)
\end{aligned}
$$
Раскроем получившееся выражение:
$$
\operatorname{var}\left(\ba \mid \bb\right)=\operatorname{var}(\bz)=\operatorname{var}\left(\ba+\bA \bb\right)
$$
$$
\begin{array}{l}
=\operatorname{var}\left(\ba\right)+\bA \operatorname{var}\left(\bb\right) \bA^{\top}+\bA \operatorname{Cov}\left(\bb, \ba\right)+\operatorname{Cov}\left(\ba, \bb\right) \bA^{\top} \\
=\Sigma_{11}+\Sigma_{12} \Sigma_{22}^{-1} \Sigma_{22} \Sigma_{22}^{-1} \Sigma_{21}-2 \Sigma_{12} \Sigma_{22}^{-1} \Sigma_{21} \\
=\Sigma_{11}+\Sigma_{12} \Sigma_{22}^{-1} \Sigma_{21}-2 \Sigma_{12} \Sigma_{22}^{-1} \Sigma_{21} \\
=\Sigma_{11}-\Sigma_{12} \Sigma_{22}^{-1} \Sigma_{21}
\end{array}
$$
Ковариации левой и правой части так же совпадают. В силу свойств нормального распределения отсюда следует искомое утверждение.
\end{proof}

\end{theorem}
Эта теорема находит множество применений, в том числе в представленной работе. Из правила Маферона следует важное
\begin{corollary}
Пусть $f \sim GP(0, k)$, $\bZ$ - множество точек размера $m$ и $f_m = f(\bZ)$. Тогда 
\begin{equation}
    \underbrace{(f \mid f_m = \boldsymbol{u})(\cdot)}_{\text {posterior }} \stackrel{\mathrm{d}}{=} \underbrace{f(\cdot)}_{\text {prior }}+\underbrace{k(\cdot, \mathbf{Z}) \mathbf{K}_{m, m}^{-1}\left(\boldsymbol{u}-\boldsymbol{f}_{m}\right)}_{\text {update }}
\end{equation}


\end{corollary}


\end{document}

